\documentclass[letterpaper,12pt]{article}

\title{Statify}

\begin{document}

\maketitle

Because the symbolic path traversal may lead to state explosion and too much needed user interaction which defeats the purpose of having an almost plug-n-play tool at the start of a security review. We will try to ease out a bit on symbolic execution.

The aim is to take some concrete representations:

% <img src=":/f2c90d0d6f4044f793c33abfb84ef362" alt="drawing" width="500"/>

And to approximate these concrete trajectories into an abstract one:

% <img src=":/bbea8b7833c84db4808f3686d3290169" alt="drawing" width="400"/>
% <img src=":/9d2be19e75184d55a216ff5f711c3cee" alt="drawing" width="400"/>

% https://www.di.ens.fr/~cousot/publications.www/video-public/Cousot_CdF_2008_02_22/Cousot_CdF_2008_02_22.mp4

Coverage-guided fuzzing
Stop when 100\% or maximum iterations 

If we meet the same JUMPDEST multiple times, write bounds. Do some BMC

Fuzzing will do the arrows in the diagram.
Find entry points of the program, also find end points.
End is where the arrow will start, entry is where the arrow goes.

% ![ca0e990be1b2ab2cfa5dd7f4a56469aa.png](:/792c30f03a694536a7a88c9dd52e1f56)

Can we assume that the execution is complete when coverage = 100% ? Probably not, if we consider symbolic jumping. Let's take this program for instance

% ```evm
% PUSH0 CALLDATALOAD PUSH1 0xFF AND
% JUMP
% JUMPDEST
% RETURN
% JUMPDEST
% RETURN
% JUMPDEST
% RETURN
% ```

Well if if have only visited one of those, the coverage wouldn't be 100%, so it sounds like a pretty good metric.

We could use abstract representation to replace concrete fuzzed inputs for jump conditions by a formula by checking if it fits all the trajectories.

\section{CFG}

When we explore the CFG, only an extremely rudimentary analysis is done.
For instance, we won't resolve `PUSH` operations when converting the bytecode to mnemonics, because this will get done later in the dynamic execution.
That means that we won't even conduct any dead code analysis during that time, we just want to extract the blocks of the cfg, starting with a `JUMPDEST` and ending either with a `JUMP/JUMPI` or a terminating block.
All of these blocks should just be stored in a `Vec<Block>`.

\section{Linking}

The linking process aim is to create a directed graph out of the flattened cfg through fuzzing.
It would be great if the fuzzing was coverage guided.
For now, we can just stop once the coverage is 100% or a limit has been reached.

Automatized state machine breaking with `proptest` https://proptest-rs.github.io/proptest/proptest/state-machine.html

When ABI is provided, we should guide fuzzing with it. Also try to send some wrong inputs.
\end{document}
